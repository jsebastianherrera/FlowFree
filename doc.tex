\documentclass[letter]{article}
\usepackage[spanish]{babel}
\usepackage[margin=1in]{geometry}
\usepackage{amsmath}
\usepackage{amsthm}
\usepackage{amssymb}
\usepackage[utf8]{inputenc}
\usepackage{graphicx, color}
\usepackage{algorithm}
\usepackage{algpseudocode}
\usepackage{mathrsfs}


% Some definitions
\floatname{algorithm}{Algoritmo}

% Author info
\title {Flow Free}
\author{ Jorge Luis Esposito Albornoz$^1$  Juan Sebastián Herrera Guaitero$^1$}
\date{
	$^1$Departamento de Ingeniería de Sistemas, Pontificia Universidad Javeriana\\Bogotá,  Colombia \\
	\texttt{\{jesposito,jsebastianherrera\}@javeriana.edu.co}\\~\\
	\today
}

\begin{document}
\maketitle

\begin{abstract}
	En este documento se presenta la documentaci\'on correspondiente a la implementaci\'on desarrollada para realizar el videojuego FlowFree.
	\textbf{Palabras clave:} Flow, puzzle.
\end{abstract}

\tableofcontents

\newpage
\section{Formalización del problema}
Un rompecabezas o puzzle  es un juego cuyo objetivo es formar una figura combinando correctamente las partes de esta, que se encuentran en distintos pedazos o en nuestro caso en particual que se compone de diferentes ``Flows''.

\subsection{Definici\'on del problema}
El juego presenta rompecabezas de enlaces numéricos. Cada rompecabezas tiene una cuadrícula de cuadrados con pares de puntos de colores que ocupan algunos de los cuadrados.
El objetivo es conectar puntos del mismo color dibujando ``tuber\'ias'' entre ellos de forma que toda la cuadrícula esté ocupada por tuberías. Sin embargo, las tuberías no pueden cruzarse. La dificultad viene determinada principalmente por el tamaño de la cuadrícula.\\\\
Flow free se define apartir de:
\begin{enumerate}
	\item Dada un matriz de $M$ de elementos $a \in \mathbb{Z} $, donde cada elemento representa un color en la cuadr\'icula .
\end{enumerate}
Permita a trav\'es de la consola completar los ``flows'' o ``tuberias'' teniendo en cuenta las reglas anteriormente descritas.
\begin{itemize}
	\item \textbf{Entradas:}
	      \begin{itemize}
		      \item $M^{nxn} ~ | ~ \forall M_{ij} \in \mathbb{Z} | \exists $
	      \end{itemize}
	\item \textbf{Salida:}
	      \begin{itemize}
		      \item $M^{nxn} ~ | ~ \forall M_{ij}, M_{ij} \neq 0 $
	      \end{itemize}
\end{itemize}
\section{Implementaci\'on}
\subsection{Modelado de datos}
En esta secci\'on se presetar\'a las estructuras de datos utilizadas para la soluci\'on.
\subsubsection{Diagrama de clases}
\begin{figure}[H]
	\centerline{\includegraphics[scale=0.50]{Class Diagram.jpg}}
	\caption{Diagrama de clases}
	\label{fig}
\end{figure}
\newpage
\subsubsection{Diagrama de secuencia}
\vspace{1cm}
\begin{figure}[H]
	\centerline{\includegraphics[scale=0.40]{sequence.jpg}}
	\caption{Diagrama de secuencia}
	\label{fig}
\end{figure}
\vspace{1cm}
\section{C\'omo jugar?}
\subsubsection{Keybinding}
\begin{center}
	\begin{tabular}{ |c|c| }
		\hline
		\textbf{Tecla/Teclas} & \textbf{Funciones}                                                \\
		\hline
		Flecha izquierda      & Moverse una columna hacia la izquierda                            \\
		\hline
		Flecha derecha        & Moverse una columna hacia la derecha                              \\
		\hline
		Flecha arriba         & Moverse una fila hacia arriba                                     \\
		\hline
		Flecha abajo          & Moverse una fila hacia abajo                                      \\
		\hline
		Enter                 & Seleccionar el color actual(solo funciona con puntos principales) \\
		\hline
		Ctrl + c              & Enviar un signal para volver al menur principal)                  \\
		\hline
	\end{tabular}
\end{center}
\subsubsection{Protocolo}
\begin{enumerate}
	\item Dirigase a la ruta en donde se encuentra el proyecto con nombre FlowFree.
	\item Asegurese de tener python3 instalado y contar con las siguiente libreria :
	      \begin{itemize}
		      \item \textbf{Linux o MacOs:}
		            \begin{itemize}
			            \item El paquete ya viene instalado por default.
		            \end{itemize}
		      \item \textbf{Windows:}
		            \begin{itemize}
			            \item windows-curses
		            \end{itemize}
	      \end{itemize}
	      En caso de no tenerla, instalarla de la siguiente forma: $$
		      pip ~ install ~  ``nombre\_paquete''$$
	\item Ejecutar el siguiente comando: $$ python3 ~ main.py$$
	      \begin{figure}[H]
		      \centerline{\includegraphics[scale=0.40]{menu.jpg}}
		      \caption{Menu principal}
		      \label{fig}
	      \end{figure}
	      Moverse de acuerdo a las Keybindings para elegir una opci\'on con la tecla ``enter''.
	      \begin{itemize}
		      \item \textbf{play}: Se gener\'a un n\'umero aleatorio para elegir un archivo del 1-22.
		            \begin{figure}[H]
			            \centerline{\includegraphics[scale=0.40]{map.jpg}}
			            \caption{Mapa}
			            \label{fig}
		            \end{figure}
		            \begin{itemize}
			            \item En la parte superior izquierda de la \textbf{Figura 3} se muestra las coordenanas actuales en donde se encuentra $(0,0)$.
			            \item En la parte superior derecha de la \textbf{Figura 3} se muestra el n\'umero de ``flows'' que han sido completados.
		            \end{itemize}
		            \begin{enumerate}
			            \item[1.] Al moverse de acuerdo a las Keybindings si se para en la posici\'on de alg\'un punto que este pintado de un color diferente al blanco y da ``enter'' el color se selecciona.
				            \begin{figure}[H]
					            \centerline{\includegraphics[scale=0.30]{chosen.jpg}}
					            \caption{Color elegido}
					            \label{fig}
				            \end{figure}
			            \item[2.] Una v\'ez elegido el color en la esquina superior izquierda de la pantalla de acuerdo a la \textbf{Figura 4} se mostrar\'a el color seleccionado.\\
				            \begin{enumerate}
					            \item[2.1] El usuario va poder moverse de acuerdo a las Keybidings y se pintar\'an los cuadros en la cuadr\'icula con el color seleccionado.
					            \item[2.2] Para completar un ``flow'', el usuario debe moverse de acuerdo a las Keybindings y conectar el ``flow'' con el punto destino, una v\'ez lo complete el programa inmediatamente deselecciona el color previamente elegido.
				            \end{enumerate}
			            \item[3. ] El juego solo acaba cuando todos los ``flows'' est\'en completados y la cuadr\'icula no tenga ning\'un espacio vacio.
			            \item[4. ] Tenga en cuenta que usted puede volver al menu principal en cualquier momento con la combinaci\'on Ctrl + c.


		            \end{enumerate}
		      \item \textbf{exit}: El proceso se termina.
	      \end{itemize}


\end{enumerate}
\end{document}

